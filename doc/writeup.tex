\documentclass[11pt]{article}
\usepackage{graphicx}
\usepackage{amsmath}
\usepackage{moreverb}
\usepackage{url}
\usepackage{verbatim}

\begin{document} 

\title{{11-761 Language and Statistics\\Spring 2012\\Course Project}}
\author{Ryan Carlson, Naoki Orii, Peter Schulam}
\maketitle

\section{Description of the Toolkit}

\section{Contributions}

\subsection{N-gram Models}

To capture short-range dependencies between part of speech tags, we use
eight different N-gram models (unigram, bigram, trigram, etc.). The
parameters are estimated using maximum likelihood with pseudo counts to
avoid assigning N-grams zero probabilities. In addition to the standard
N-gram models, we also added unigram and bigram features to our maximum
entropy model. 

\subsection{Triggers}
N-grams can not capture long distance information.
For example, if we have observed a left parenthesis in a given sentence,
there is a highly likelihood that we will observe a right parenthesis in the same sentence,
but n-grams will fail to predict this.
We capture this long distance information by adding triggers pairs as feature functions.
To formulate a trigger pair $A \rightarrow B$ as a constraint, we define the feature function $f_{A \rightarrow B}$ as:
\[
  f_{A \rightarrow B}(h, w) = \begin{cases}
    1 & (\textrm{if } A \in h \textrm{ and } w = B) \\
    0 & (otherwise)
  \end{cases}
\]
where $h$ and $w$ denote the history and the word, respectively.

Using the training data, we computed the average mutual information for the 1089 possible triggers pairs.
In Table \ref{t:triggerpairs}, we list trigger pairs and their corresponding mutual information (MI) values, sorted by decreasing order of MI.

\begin{table*}[h!]
\begin{small}
\begin{center}
\caption{Trigger {\sf A} for word {\sf B}, sorted by MI in decreasing order}
\label{t:triggerpairs}
\begin{tabular}{|l|l|l|}
\hline
{\sf A} & {\sf B} & {\sf Mutual Information} \\
\hline
CD & CD & 0.00933 \\
$<$LEFTPAR$>$ & $<$RIGHTPAR$>$ & 0.00443 \\
$<$PERIOD$>$ & $<$PERIOD$>$ & 0.00431 \\
VBD & VBD & 0.00307 \\
NNP & NNP & 0.00302 \\
VBZ & CD & 0.00279 \\
PRP & CD & 0.00259 \\
$<$COLON$>$ & $<$COLON$>$ & 0.00248 \\
VB & CD & 0.00233 \\
VBZ & VBD & 0.00226 \\
VBP & CD & 0.00196 \\
VBD & VBZ & 0.00169 \\
PRP & PRP & 0.00151 \\
VBZ & VBZ & 0.00145 \\
VBD & VBP & 0.00144 \\
VBP & VBP & 0.00141 \\
VBP & VBD & 0.00140 \\
VBD & CD & 0.00131 \\
RB & CD & 0.00123 \\
DT & CD & 0.00113 \\
MD & CD & 0.000944 \\
... & ... & ... \\
\hline
\end{tabular}\vspace*{-5mm}
\end{center}
\end{small}
\end{table*}

\noindent
It can be seen from the table that {\em self-triggers}, or words that trigger themselves (such as CD $\rightarrow$ CD) comprise the majority of the top 10 trigger pairs.
As expected, we see that $<$LEFTPAR$>$ $\rightarrow$ $<$RIGHTPAR$>$ has a high mutual information.
Similar to Rosenfeld \cite{rosenfeld1996}, we only incorporated pairs that had at least 0.001 bit of average mutual information into our system.


\subsection{Long Distance N-grams}

Long distance $N$-grams are extensions of $N$-grams where a word is predicted from $N-1$-grams some distance back in the history.
For example, a distance-2 bigram predicts $w_i$ from the history $w_{i-2}$.
Constraints for distance-$j$ bigram $\{w_1, w_2\}$ can be formulated as follows:
\[
  f_{\{w_1, w_2\}}^{j}(h, w) = \begin{cases}
    1 & (\textrm{if } w_{i-j} = w_1 \textrm{ and } w = w_2) \\
    0 & (otherwise)
  \end{cases}
\]

\noindent
In our system, we considered distance-2 bigrams and distance-2 trigrams.

\subsection{Shallow Parse Tree Constituents}

To introduce aspects of syntax into our toolkit, we defined a number of
binary valued ``chunk rule'' features for our maximum entropy model. The
general idea is to use a number of production rules taken from a
context-free grammar to check for higher level syntactic structure. For
example, suppose that the current history of characters in the token
stream is \texttt{DT NN VB NN PP DT}, and we want to know the
probability that the next token is \texttt{NN}. If we have the 
production rule below defined in our shallow grammar file, then the
feature \texttt{feature\_PP} will be true when calculating the
probability \texttt{P(NN|DT NN VB NN PP DT)}.

\begin{verbatim}
PP -> PP DT NN
\end{verbatim}

We gathered chunk rules by first going through a context free grammar
for English that we used in another class, and cleaning the part of
speech tag set used in the grammar to match the ones in our tag set.
Since there were some tags that did not have any analogues in our set,
we then kept only those production rules for which the right hand side
of the rule contained only tags that were in our tag set. This removed
most of the ``high level'' tags that can only be built higher in a parse
tree, which is why we call these features ``shallow'' features.

\subsection{Checking For Subset Existence}

In order to get a natural sense of higher-level parts of speech (e.g., nouns, verbs, punctuation) we grouped the tags into the following lists:

\begin{verbatim}
[JJ, JJR, JJS] # adjectives
[NN, NNS, NNP, NNPS] # nouns
[PRP, PRP$] # pronouns
[RB, RBR] # adverbs
[VB, VBD, VBG, VBN, VBP, VBZ] # verbs
[WDT, WP, WRB] # wh words
[<COLON>, <COMMA>, <LEFTPAR>, <PERIOD>, <RIGHTPAR>] # punctuation
[CC] # coordinating conjunction
[CD] # cardinal number
[DT] # determiner
[EX] # existential THERE
[IN] # preposition or subordinating conjunction
[MD] # modal
[POS] # possessive ending
[RP] # particle
[TO] # TO
[CC, CD, DT, EX, IN, MD, POS, RP, TO] #other
\end{verbatim}

Each of the seven major parts of speech groups are represented, and then each of the rest are in both a singleton group and in a special ``other'' group. We do this because the nine singletons do not really group together in any linguistically meaningful way, so we let the maxent model decide whether or not an ``other'' category has more predictive power than each of the features on their own. We could have taken this a step further and allowed any combination of the features to group together, but this would have both become too computationally expensive and would not have had the linguistic power we wanted.

Now, for each group $G$, a lookback number $i$, and a history \[ H = \{h_1, h_2, \dots, h_n\},\] we check if
\[ \forall g \in G,\; g \in \{h_{n}, h_{n-1}, \ldots, h_{n-i+1}\}  \]
For example, suppose $G$ was the group of adjectives (i.e., $g$ could take the value \texttt{JJ}, \texttt{JR}, or \texttt{JJS}) and that the lookback number $i = 2$. Furthermore, suppose the history was \texttt{\{<PERIOD>, DT, JJ, NN\}}. Then the adjective \texttt{JJ} would fire (evaluate to \texttt{True}) because it matches the second to last word in the history. However, if the target group $G$ were determiners, the feature would not fire because the determiner \texttt{DT} is too far in the past.

\section{Comments and Suggestions}


\begin{thebibliography}{}

\bibitem{rosenfeld1996} 
R. Rosenfeld,
``A Maximum Entropy Approach to Adaptive Statistical Language Modeling,''
{\em Computer, Speech, and Language}, vol. 10, pp.187-228, 1996.

\end{thebibliography}

\end{document}
